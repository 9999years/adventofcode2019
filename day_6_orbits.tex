\typeout{^^J%
  Solution code for Advent of Code, Day 6: Universal Orbit Map ^^J%
  Implemented in LaTeX3; see https://www.ctan.org/pkg/l3kernel ^^J%
  https://adventofcode.com/2019/day/6 ^^J%
}
\errorcontextlines 10
\input{regression-test.tex}
\RequirePackage{expl3}
\ExplSyntaxOn

% \aoc_parse_orbit:Nn <tlist> <orbit spec>
% Splits <orbit spec> by ")" and stores the two halves in <tlist> as two
% unbraced tokens.
\seq_new:N \l__aoc_po_tmp_seq
\tl_new:N \l__aoc_po_tmpa_tl
\tl_new:N \l__aoc_po_tmpb_tl
\cs_set:Npn \aoc_parse_orbit:Nn #1#2
  {
    \seq_set_split:Nnn \l__aoc_po_tmp_seq { ) } { #2 }
    \seq_get_left:NN \l__aoc_po_tmp_seq \l__aoc_po_tmpa_tl
    \seq_get_right:NN \l__aoc_po_tmp_seq \l__aoc_po_tmpb_tl
    \tl_set:Nx #1
      {
        { \tl_use:N \l__aoc_po_tmpb_tl }
        \tl_use:N \l__aoc_po_tmpa_tl
      }
  }

% \aoc_parse_orbit_map_to_seq:NN <output seq> <input seq>
% Maps <input seq> of orbit specs into <output seq> according to
% \aoc_parse_orbit:Nn.
\tl_new:N \l__aoc_pom_tmp_tl
\cs_set:Npn \aoc_parse_orbit_map_to_seq:NN #1#2
  {
    \seq_clear:N #1
    \seq_map_inline:Nn #2
      {
        \aoc_parse_orbit:Nn \l__aoc_pom_tmp_tl { ##1 }
        \seq_put_right:Nx #1 { \tl_use:N \l__aoc_pom_tmp_tl }
      }
  }

% \aoc_read_to_seq:Nn <sequence> <filename>
% Reads the file given by <filename> and stores it to <sequence>,
% splitting by lines. Caveats about matching brace-groups apply, but
% they're not relevant here.
\ior_new:N \l__aoc_rts_stream
\cs_set:Npn \aoc_read_to_seq:Nn #1#2
  {
    \ior_open:Nn \l__aoc_rts_stream { #2 }
    \seq_clear:N #1
    \ior_map_inline:Nn \l__aoc_rts_stream
      {
        \seq_put_right:Nn #1 { ##1 }
      }
    \ior_close:N \l__aoc_rts_stream
  }

% The orbit map datatype stores a directed graph of orbits (e.g. X orbits Y,
% Y orbits Z, etc.) as a key-value mapping. If a mapping X->Y is in an orbit
% map, we interpret that as "X is in orbit about Y".
%
% Each object orbits exactly one other, except for the universal Center of
% Mass (COM). However, each object can *be orbited by* more than one object,
% and these queries are often the more interesting ones.
%
% The number of direct orbits is the size of the orbit map.
%
% The number of indirect orbits is a bit trickier.
%   - Store a map of visited nodes called 'V'.
%     - V maps a node to its number of indirect orbits.
%   - Store a set of unvisited nodes called 'U'.
%   - While U is not empty:
%     - Pop some (u, a) from U.
%       - 'u' is in orbit around 'a'.
%     - If 'a' is not in V:
%       - Recursively visit 'a', adding its orbits to 'u'.

% \om_new:N creates a new, empty orbit map variable. The declaration is made
% globally.
\cs_set_eq:NN \om_new:N \prop_new:N

% \om_direct_orbits:N <orbit map>
% Returns the number of direct orbits in the map.
\cs_set_eq:NN \om_direct_orbits:N \prop_count:N

% \om_parse_file:Nn <orbit map> <filename>
% Reads <filename> and parses it as an orbit map, storing it in <orbit map>.
\seq_new:N \l__om_parse_tmpa_seq
\seq_new:N \l__om_parse_tmpb_seq
\tl_new:N \l__om_parse_tmpa_tl
\cs_set:Npn \om_parse_file:Nn #1#2
  {
    \prop_clear:N #1
    \aoc_read_to_seq:Nn \l__om_parse_tmpa_seq { #2 }
    \aoc_parse_orbit_map_to_seq:NN \l__om_parse_tmpb_seq \l__om_parse_tmpa_seq
    \seq_map_variable:NNn \l__om_parse_tmpb_seq \l__om_parse_tmpb_tl
      {
        \tl_set:Nx \l_tmpa_tl { \tl_head:N \l__om_parse_tmpb_tl }
        \tl_set:Nx \l_tmpb_tl { \tl_tail:N \l__om_parse_tmpb_tl }
        \prop_put:NVV #1 \l_tmpa_tl \l_tmpb_tl
      }
  }

% Internal. Visits the given node, and possibly others recursively, updating
% the unvisited map and indirect orbit counter int.
\prop_new:N \l__om_visited_prop
\prop_new:N \l__om_unvisited_prop
\int_new:N \l__om_indirect_int
\tl_new:N \l__om_orbits_around_tl
\cs_set:Npn \om__visit:n #1
  {
    \prop_get:NnNT \l__om_unvisited_prop { #1 } \l__om_orbits_around_tl
      {
        % #1 is unvisited and orbits around \l__om_orbits_around_tl
        \prop_get:NVNTF \l__om_visited_prop \l__om_orbits_around_tl \l__om_indirect_tl
          {
            % \l__om_orbits_around_tl is visited.
            % Set Visited[#1] to 1 + \l__om_indirect_tl
            \tl_set:Nx \l_tmpa_tl
              {
                \int_eval:n { 1 + \tl_use:N \l__om_indirect_tl }
              }
            \prop_put:NnV \l__om_visited_prop { #1 } \l_tmpa_tl
            \int_add:Nn \l__om_indirect_int { \tl_use:N \l_tmpa_tl }
            \prop_remove:Nn \l__om_unvisited_prop { #1 }
          }
          {
            % \l__om_orbits_around_tl is not visited.
            % Recursively compute Visited[##2]
            \om__visit:x { \tl_use:N \l__om_orbits_around_tl }
            \om__visit:n { #1 }
          }
      }
  }
\cs_generate_variant:Nn \om__visit:n { x }

% Internal. Processes the next unvisited node and recurses, or quits if
% empty.
\tl_new:N \l__om_indirect_tl
\cs_set:Npn \om__indirect_orbits:
  {
    \prop_map_inline:Nn \l__om_unvisited_prop
      {
        % ##1 orbits around ##2
        \om__visit:n { ##1 }
        \prop_map_break:
      }
    \prop_if_empty:NF \l__om_unvisited_prop
      { \om__indirect_orbits: }
  }

% \om_indirect_orbits:NN <int-var> <orbit-map>
% Stores the number of indirect orbits in <orbit-map> in the variable
% specified by <int-var>.
\cs_set:Npn \om_indirect_orbits:NN #1#2
  {
    \prop_clear:N \l__om_visited_prop
    % Avoid a recursive loop at the base case.
    \prop_put:Nnn \l__om_visited_prop
      { COM } { -1 }

    % Clear other values.
    \prop_set_eq:NN \l__om_unvisited_prop #2
    \int_zero:N \l__om_indirect_int

    % Process the prop lists.
    \om__indirect_orbits:

    % Set the return value.
    \int_set_eq:NN #1 \l__om_indirect_int
  }

% \om_total_orbits:NN <int-var> <orbit map>
% Returns the total number of orbits (i.e. both direct and indirect) in the map.
\cs_set:Npn \om_total_orbits:NN #1#2
  {
    \om_indirect_orbits:NN \l_tmpa_int #2
    \int_set:Nn #1
      {
        \om_direct_orbits:N #2
        + \int_use:N \l_tmpa_int
      }
  }

% Breadth-first search implementation.

% Internal: The queue of nodes to discover.
%   - Enqueue with \seq_put_left:Nx
%   - Dequeue with \seq_pop_right:NN
\seq_new:N \l__om_bfs_discovery_seq
% Internal: Discovered nodes.
\seq_new:N \l__om_bfs_discovered_seq
% Internal: Parents of discovered nodes.
\prop_new:N \l__om_bfs_parent_prop
% Internal: Undiscovered edges.
\prop_new:N \l__om_bfs_undiscovered_prop
% Internal: Goal node.
\tl_new:N \l__om_bfs_goal_tl
% Internal: Start node.
\tl_new:N \l__om_bfs_start_tl

% \om__adjacent_edges:n <node>
% Marks the adjacent and undiscovered neighbors of <nodes> as discovered,
% notes their parents, and adds them to the discovery queue.
\cs_set:Npn \om__adjacent_edges:n #1
  {
    % For each undiscovered edge...
    \prop_map_inline:Nn \l__om_bfs_undiscovered_prop
      {
        % ...if the edge starts or ends at #1...
        \str_if_eq:nnT { ##1 } { #1 }
          {
            \seq_if_in:NnF \l__om_bfs_discovered_seq { ##2 }
              {
                % Label it as discovered.
                \seq_put_left:Nn \l__om_bfs_discovered_seq { ##1 }
                \prop_remove:Nn \l__om_bfs_undiscovered_prop { ##1 }
                % Mark its parent.
                \prop_put:Nnn \l__om_bfs_parent_prop { ##2 } { ##1 }
                % Add it to the discovery queue.
                \seq_put_left:Nn \l__om_bfs_discovery_seq { ##2 }
              }
          }
        \str_if_eq:nnT { ##2 } { #1 }
          {
            % ...and the node the edge leads to is undiscovered:
            \seq_if_in:NnF \l__om_bfs_discovered_seq { ##1 }
              {
                % Label it as discovered.
                \seq_put_left:Nn \l__om_bfs_discovered_seq { ##2 }
                \prop_remove:Nn \l__om_bfs_undiscovered_prop { ##2 }
                % Mark its parent.
                \prop_put:Nnn \l__om_bfs_parent_prop { ##1 } { ##2 }
                % Add it to the discovery queue.
                \seq_put_left:Nn \l__om_bfs_discovery_seq { ##1 }
              }
          }
      }
  }
\cs_generate_variant:Nn \om__adjacent_edges:n { V }

\cs_set:Npn \om__shortest_path:
  {
    \seq_if_empty:NF \l__om_bfs_discovery_seq
      {
        \seq_pop_right:NN \l__om_bfs_discovery_seq \l_tmpa_tl
        \tl_if_eq:NNTF \l_tmpa_tl \l__om_bfs_goal_tl
          {
            % Found the goal; quit!
            \seq_clear:N \l__om_bfs_discovery_seq
          }
          {
            % Discover adjacent edges.
            \om__adjacent_edges:V \l_tmpa_tl
          }
      }
    % Recurse.
    \seq_if_empty:NF \l__om_bfs_discovery_seq { \om__shortest_path: }
  }

% \om_shortest_path:NNnn <seq var> <orbit map> <search from> <search to>
% Breadth-first search for pathfinding on an orbit map.
\cs_set:Npn \om_shortest_path:NNnn #1#2#3#4
  {
    % #1: seq var to store result path in
    % #2: orbit map
    % #3: start node (you)
    % #4: goal node (santa)
    \seq_clear:N \l__om_bfs_discovery_seq
    \seq_put_left:Nx \l__om_bfs_discovery_seq { #3 }

    \seq_clear:N \l__om_bfs_discovered_seq
    \seq_put_left:Nn \l__om_bfs_discovered_seq { #3 }

    \prop_clear:N \l__om_bfs_parent_prop

    \prop_set_eq:NN \l__om_bfs_undiscovered_prop #2

    \tl_set:Nn \l__om_bfs_goal_tl { #4 }
    \tl_set:Nn \l__om_bfs_start_tl { #3 }

    \om__shortest_path:

    % Use parent links to get back from the start to the goal.
    \seq_clear:N #1
    \seq_put_right:Nn #1 { #4 }
    \str_clear:N \l_tmpa_str
    \str_set:Nn \l_tmpa_str { #4 }
    \bool_do_until:nn
      {
        \str_if_eq_p:Vn \l_tmpa_str { #3 }
      }
      {
        \prop_get:NVN \l__om_bfs_parent_prop \l_tmpa_str \l_tmpb_tl
        \seq_put_right:NV #1 \l_tmpb_tl
        \str_set:Nx \l_tmpa_str { \tl_use:N \l_tmpb_tl }
      }
  }

% \om_min_orbital_transfers:NNnn <int var> <orbit map> <search from> <search to>
\cs_set:Npn \om_min_orbital_transfers:NNnn #1#2#3#4
  {
    \om_shortest_path:NNnn \l_tmpa_seq #2 { #3 } { #4 }
    \int_set:Nn #1 { \seq_count:N \l_tmpa_seq - 3 }
  }

\START

\tl_new:N \l_aoc_test_tl
\TEST { aoc_parse_orbit:Nn }
  {
    \aoc_parse_orbit:Nn \l_aoc_test_tl { COM)BOOK }
    \ASSERT { \tl_head:N \l_aoc_test_tl } { BOOK }
    \ASSERT { \tl_tail:N \l_aoc_test_tl } { COM }
  }

\seq_new:N \l_aoc_test_seq
\TEST { aoc_read_to_seq:Nn }
  {
    \aoc_read_to_seq:Nn \l_aoc_test_seq { data/day_6_orbit_map_test.txt }
    \ASSERT { \seq_use:Nn \l_aoc_test_seq { , } }
      {
        COM)B ,
        B)C   ,
        C)D   ,
        D)E   ,
        E)F   ,
        B)G   ,
        G)H   ,
        D)I   ,
        E)J   ,
        J)K   ,
        K)L
      }
  }

\TEST { aoc_parse_orbit_map_to_seq:NN }
  {
    \seq_new:N \l_aoc_testinput_seq
    \seq_new:N \l_aoc_testoutput_seq
    \aoc_read_to_seq:Nn \l_aoc_testinput_seq { data/day_6_orbit_map_test.txt }

    \aoc_parse_orbit_map_to_seq:NN \l_aoc_testoutput_seq \l_aoc_testinput_seq
    \ASSERT { \seq_use:Nn \l_aoc_testoutput_seq { , } }
      {
        {B}COM ,
        {C}B   ,
        {D}C   ,
        {E}D   ,
        {F}E   ,
        {G}B   ,
        {H}G   ,
        {I}D   ,
        {J}E   ,
        {K}J   ,
        {L}K
      }
  }

% All orbit-map declarations are global.
\om_new:N \l_aoc_tmp_om
\TEST { om_parse_file:Nn }
  {
    \om_parse_file:Nn \l_aoc_tmp_om { data/day_6_orbit_map_test.txt }

    \cs_set:Npn \aoc_assert_in:Nnn #1#2#3
      {
        \prop_get:NnN #1 { #2 } \l_aoc_test_tl
        \ASSERT { \tl_use:N \l_aoc_test_tl } { #3 }
      }

    \aoc_assert_in:Nnn \l_aoc_tmp_om {B} {COM}
    \aoc_assert_in:Nnn \l_aoc_tmp_om {C} {B}
    \aoc_assert_in:Nnn \l_aoc_tmp_om {D} {C}
    \aoc_assert_in:Nnn \l_aoc_tmp_om {E} {D}
    \aoc_assert_in:Nnn \l_aoc_tmp_om {F} {E}
    \aoc_assert_in:Nnn \l_aoc_tmp_om {G} {B}
    \aoc_assert_in:Nnn \l_aoc_tmp_om {H} {G}
    \aoc_assert_in:Nnn \l_aoc_tmp_om {I} {D}
    \aoc_assert_in:Nnn \l_aoc_tmp_om {J} {E}
    \aoc_assert_in:Nnn \l_aoc_tmp_om {K} {J}
    \aoc_assert_in:Nnn \l_aoc_tmp_om {L} {K}
  }

\TEST { om_direct_orbits:N }
  {
    \om_parse_file:Nn \l_aoc_tmp_om { data/day_6_orbit_map_test.txt }
    \ASSERT { \om_direct_orbits:N \l_aoc_tmp_om } { 11 }
  }

\int_new:N \l_aoc_test_int
\TEST { om_indirect_orbits:NN }
  {
    \om_parse_file:Nn \l_aoc_tmp_om { data/day_6_orbit_map_test.txt }
    \om_indirect_orbits:NN \l_aoc_test_int \l_aoc_tmp_om
    \ASSERT { \int_use:N \l_aoc_test_int } { 31 }
  }

\TEST { om_total_orbits:NN }
  {
    \om_parse_file:Nn \l_aoc_tmp_om { data/day_6_orbit_map_test.txt }
    \om_total_orbits:NN \l_aoc_test_int \l_aoc_tmp_om
    \ASSERT { \int_use:N \l_aoc_test_int } { 42 }
  }

\TEST { Day~6~Part~1 }
  {
    \om_parse_file:Nn \l_aoc_tmp_om { data/day_6_orbit_map.txt }
    \om_total_orbits:NN \l_aoc_test_int \l_aoc_tmp_om
    \ASSERT { \int_use:N \l_aoc_test_int } { 186 597 }
  }

\TEST { BFS }
  {
    \om_parse_file:Nn \l_aoc_tmp_om { data/day_6_orbit_map_test_2.txt }
    \om_shortest_path:NNnn \l_aoc_test_seq \l_aoc_tmp_om
      { YOU } { SAN }
    \ASSERTSTR { \seq_use:Nn \l_aoc_test_seq { , } }
      { SAN
      , I
      , D
      , E
      , J
      , K
      , YOU
      }

      \om_min_orbital_transfers:NNnn \l_tmpb_int \l_aoc_tmp_om { YOU } { SAN }
      \ASSERT { \int_use:N \l_tmpb_int } { 4 }
  }

\TEST { Day~6~Part~2 }
  {
    \om_parse_file:Nn \l_aoc_tmp_om { data/day_6_orbit_map.txt }
    \om_min_orbital_transfers:NNnn \l_tmpb_int \l_aoc_tmp_om { YOU } { SAN }
    \ASSERT { \int_use:N \l_tmpb_int } { 412 }
  }

\END
\batchmode
